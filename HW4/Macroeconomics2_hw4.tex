%各自の環境に応じて修正
%\documentclass[a4paper,11pt]{jarticle}
\documentclass{ltjsarticle}

\usepackage{enumitem}
\usepackage{appendix}
\usepackage{float}
\usepackage{bm}
\usepackage[dvipdfmx]{graphicx}
\usepackage[dvipdfmx]{color}
\usepackage{tikz}
\usepackage[T1]{fontenc}
\usepackage[utf8]{inputenc}
\usepackage{lmodern}
\usepackage[american]{babel}
\usepackage{subcaption}
\usepackage{tabularx}
\usepackage{graphics}
\usepackage{physics}
\usepackage{mathtools}
\usepackage{amssymb,amsmath,amsfonts,eurosym,geometry,graphicx,caption,color,setspace,comment,footmisc,caption,pdflscape,array,hyperref}
\usepackage{booktabs}
\usepackage{siunitx}
\newcolumntype{d}{S[
    input-open-uncertainty=,
    input-close-uncertainty=,
    parse-numbers = false,
    table-align-text-pre=false,
    table-align-text-post=false
]}
\usepackage{listings} 
\usepackage{xcolor}
\lstdefinestyle{matlab}{
  language=Matlab,
  basicstyle=\ttfamily\footnotesize,
  keywordstyle=\color{blue},
  stringstyle=\color{red},
  commentstyle=\color{green!60!black},
  numbers=left,
  numberstyle=\tiny\color{gray},
  stepnumber=1,
  frame=single,
  breaklines=true,
  showstringspaces=false
}

 
\usepackage{titlesec}
\titleformat*{\section}{\Large\rmfamily}
\titleformat*{\subsection}{\large\rmfamily}
\titleformat*{\subsubsection}{\rmfamily}
 

%テキストの表示領域の調節
\setlength{\textwidth}{\paperwidth}
\addtolength{\textwidth}{-40truemm}
\setlength{\textheight}{\paperheight}
\addtolength{\textheight}{-45truemm}

%余白の調節
\setlength{\topmargin}{-10.4truemm}
\setlength{\evensidemargin}{-5.4truemm}
\setlength{\oddsidemargin}{-5.4truemm}
\setlength{\headheight}{17pt}
\setlength{\headsep}{10mm}
\addtolength{\headsep}{-17pt}
\setlength{\footskip}{5mm}

% \renewcommand{\thesection}{\Alph{section}}
% \renewcommand{\thesubsection}{\alph{subsection}}


\title{Macroeconomics $\mathrm{II}$ Homework 3}
\date{\today}
\author{Graduate School of Economics, The University of Tokyo\\[4mm]29--246029 Rin NITTA\\ 29-246033 Rei HANARI \\ 29--246004 Kosuke IGARASHI}

\begin{document}
\maketitle
\section*{Q1}
\subsection*{(a)}


The recursive formulation of a standard neoclassical growth
model studied in class in Lecture 3 is
\begin{align*}
    v(k)=\max_{0\leq k'\leq f(k)} \qty{U(f(k)-k')+\beta v(k')}.
\end{align*}

\section*{Q2}

\section*{Q3}

\section*{Q4}

\subsection*{(a)}

Vacancies are filled at rate
\begin{gather*}
  \frac{m(u,v)}{v} = \frac{u}{v+u} = \frac{\frac{1}{\theta}}{1+\frac{1}{\theta}} = m \left(\frac{1}{\theta}, 1\right) \equiv q(\theta)
\end{gather*}
As we did in the lecture, suppose there is a continuum of workers with measure 1. Every period, $\lambda(1-u)$ workers enter unemployment, and $\theta q(\theta) u $ workers find a job.
\begin{gather*}
  \Delta u = \lambda(1-u) + \theta q(\theta) u
\end{gather*}
At the steady state, $\Delta u = 0$, so we have
\begin{align*}
  u 
  &= \frac{\lambda}{\lambda + \theta q(\theta)}\\
  &= \frac{\lambda}{\lambda + \frac{1}{1+\frac{1}{\theta}}}\\
  &= \frac{\lambda}{\lambda + \frac{\theta}{\theta+1}}
\end{align*}

\subsection*{(b)}
At any period of steady state, $1-u$ workers are employed. For each employed worker, the probability of transitioning from employment to unemployment is $\lambda$, and the probability of being unemployed for $n$ periods is $(1- \theta q(\theta))^n$. Therefore, for each $n=1,2,\cdots$, 
\begin{gather*}
  u_n = \lambda (1-u) (1- \theta q(\theta))^n
\end{gather*}
\begin{align*}
  \sum_{n=1}^{\infty} u_n
  &= \frac{\lambda (1-u)}{1 - (1- \theta q(\theta))}\\
  &= \frac{\lambda \left(1-\frac{\lambda}{\lambda + \frac{\theta}{\theta+1}}\right)}{\theta q(\theta)}\\
  &= \frac{\lambda}{\lambda + \frac{\theta}{\theta+1}}\\
  &= u
\end{align*}

\subsection*{(c)}

\subsection*{(d)}

\begin{align*}
  V &= -pc + \beta \{q(\theta)J + [1-q(\theta)]V \}\\
  J &= p - w + \beta[\lambda V + (1-\lambda)J]
\end{align*}

\subsection*{(e)}

Workers' states are classified in 3 states: employed, unemployed for 1 period, and unemployed for 2 or more periods. 
Let $U_1$ be the value of unemployment for 1 period, $U_2$ be the value of unemployment for 2 or more periods, and $W$ be the value of employment. Then,
\begin{align*}
  U_1 &= z + \beta \left\{ \theta q(\theta) W + [1-\theta q(\theta)] U_2 \right\}\\
  U_2 &= 0 + \beta \left\{ \theta q(\theta) W + [1-\theta q(\theta)] U_2 \right\}\\
  W &= w + \beta \left\{\lambda U_1 + (1-\lambda) W \right\}
\end{align*}

\subsection*{(f)}

Free entry implies $V=0$. Then, solving the equations in (d) gives
\begin{gather*}
  p = w + \frac{\left(\frac{1-\beta}{\beta} + \lambda\right)pc}{q(\theta)} =  w + \left(\frac{1-\beta}{\beta} + \lambda\right)pc (\theta+1)
\end{gather*}
Post vacancies up to the point where marginal product $p$ equals wage $w$ plus expected capitalizes value of hiring costs.

\end{document}